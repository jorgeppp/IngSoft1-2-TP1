\documentclass[a4paper]{article}
\usepackage[spanish]{babel}
\usepackage[utf8]{inputenc}
\usepackage{fancyhdr}
%\usepackage{charter}   % tipografía
\usepackage{graphicx}
\usepackage{makeidx}
\usepackage{mathptmx}

\usepackage{float}
\usepackage{amsmath, amsthm, amssymb}
\usepackage{amsfonts}
\usepackage{sectsty}
\usepackage{wrapfig}
\usepackage{listings} % necesario para el resaltado de sintaxis
\usepackage{caption}
%%%%%%%%%% Paquete para hacer grafos
%%% ver link http://www.texample.net/tikz/examples/bridges-of-konigsberg/
%\usepackage{fullpage}
%\usepackage{fourier}
\usepackage{tikz}
\usetikzlibrary{arrows,%
                shapes,positioning}
                
\thispagestyle{empty}
%%%%%%%%%% Fin paquete para hacer grafos
%%%%%%%%%%
\usepackage{hyperref} % agrega hipervínculos en cada entrada del índice
\hypersetup{          % (en el pdf)
    colorlinks=true,
    linktoc=all,
    citecolor=black,
    filecolor=black,
    linkcolor=black,
    urlcolor=black
}

\input{codesnippet}
\input{page.layout}
\usepackage{underscore}
\usepackage{caratula}
\usepackage{url}
\usepackage{color}
\usepackage{clrscode3e} % necesario para el pseudocodigo (estilo Cormen)

%\usepackage{algorithm}
%\usepackage{algorithmic}
\usepackage{algorithm}[1]
\usepackage{algorithmic}[1]
%\usepackage{algpseudocode}

\definecolor{dkgreen}{rgb}{0,0.6,0}
\definecolor{gray}{rgb}{0.5,0.5,0.5}
\definecolor{mauve}{rgb}{0.58,0,0.82}

\definecolor{gray}{gray}{0.5}
\definecolor{light-gray}{gray}{1}
\definecolor{orange}{rgb}{1,0.5,0}

\lstset{frame=tb,
  language=C++,
  aboveskip=3mm,
  belowskip=3mm,
  showstringspaces=false,
  columns=flexible,
  basicstyle={\small\ttfamily},
  keywordstyle=\color{blue},
  commentstyle=\color{gray},
  stringstyle=\color{mauve},
  breaklines=true,
  breakatwhitespace=true,
  tabsize=3,
  numbers=left,
  xleftmargin=2em,
  frame=single,
  framexleftmargin=2em,
  numbersep=5pt,                   % how far the line-numbers are from the code
  numberstyle=\small\color{gray} % the style that is used for the line-numbers
 }
 
 \lstdefinestyle{customc}{
  backgroundcolor=\color{light-gray},
  belowcaptionskip=1\baselineskip,
  breaklines=true,
  numbers=left,
  xleftmargin=\parindent,
  language=C++,
  showstringspaces=false,
  basicstyle=\footnotesize\ttfamily,
  keywordstyle=\bfseries\color{blue},
  commentstyle=\itshape\color{gray},
  identifierstyle=\color{black},
  stringstyle=\color{orange},
}

% Entorno dentro del cual se declaran las stories del product backlog con el
% macro \story.
\newenvironment{stories}{
  \begin{itemize}
}{
  \end{itemize}
}

% Uso: \story{rol}{historia}{criterio de aceptación}.
\newcommand{\story}[3]{
  \item
   \textbf{Como} #1 \textbf{quiero} #2 \textbf{para} #3.
}

\begin{document}


\thispagestyle{empty}
\materia{Ingeniería del Software I \& II}
\submateria{Primer Cuatrimestre de 2017}
\titulo{Trabajo Práctico 1: SimOil}
%\subtitulo{SOScrabel}
\integrante{Jorge Porto}{376/11}{cuanto.p.p@gmail.com} % por cada integrante (apellido, nombre) (n° libreta) (e-mail)

\maketitle
\newpage

\thispagestyle{empty}
\vfill

\thispagestyle{empty}
\vspace{1.5cm}
\tableofcontents
\newpage

%\normalsize

\newpage
\section{Introducción}
\setcounter{page}{1}
%\input{introduccion}
%\newpage

%\newpage

\section{Especificación de requerimientos con user stories}
\subsection{Roles de usuario}
Identificamos como rol de usuario a "Ministerio de Energía", que utilizará la aplicación con fines específicos.
\subsection{Product backlog}

El product backlog del proyecto se compone de la siguiente user storie:
\begin{stories}
  \story{Ministerio de energía}
        {simular la extracción de petróleo}
        {obtener información que me permita estimar el canon a ser cobrado en la licitación}

  \medskip
\textbf{Descripción}

\medskip
\textbf{Criterio de aceptación}

\begin{itemize}
  \item[$\circ$] En la simulación se debe obtener el loggeo diario  de actividad, el costo total del emprendimiento, el dinero total vendido de gas y petróleo, y la ganancia resultante.
 
  \end{itemize}

  \medskip
\textbf{Tareas}




  \medskip
\textbf{Sub-Stories}
\begin{stories}
\story{Ministerio de energía}
        {ingresar los parámetros de simulación}
        {definir un marco para la simulación}
        
        \medskip
\textbf{Descripción}

\medskip
\textbf{Criterio de aceptación}

\begin{itemize}
  \item[$\circ$] En
 
  \end{itemize}

  \medskip
\textbf{Tareas}  
        
\end{stories}

\end{stories}





%\input{apendice}



%\vspace*{0.5cm}

%\begin{lstlisting}
%int main(){
%  return 0;
%}
%\end{lstlisting}


%\vspace*{0.5cm}

%\newpage
%\subsection{Código del Problema 3}

%\begin{lstlisting}
%int main(){
%  return 0;
%}
%\end{lstlisting}

%\vspace*{0.5cm}

%\begin{lstlisting}
%int main(){
%  return 0;
%}
%\end{lstlisting}




\end{document}
